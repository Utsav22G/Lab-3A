\documentclass{article}
\usepackage[utf8]{inputenc}
\usepackage{graphicx}
\usepackage{latexsym}
\usepackage{listings}
\usepackage{xcolor}
\usepackage{mathtools}
\usepackage{amssymb}
\usepackage{gensymb}
\usepackage{subfig}
\documentclass{article}
\usepackage[utf8]{inputenc}
\usepackage{graphicx}
\usepackage{latexsym}
\usepackage{listings}
\usepackage{xcolor}
\usepackage{mathtools}
\usepackage{amssymb}
\usepackage{gensymb}
\usepackage{subfig}
\usepackage{mathrsfs}
\usepackage{amsmath}
\usepackage{biblatex}
\addbibresource{bib.bib}
\usepackage{float}
\usepackage{minted}
\definecolor{bg}{rgb}{0.95,0.95,0.95}


\title{Principles of Wireless Communications Lab 3A}
\author{Kristtiya Guerra, Utsav Gupta} %last name alphabetical
\date{\today}

\begin{document}

\maketitle

\begin{abstract}
    Orthogonal Frequency Division Multiplexing (OFDM) is a modulation technique that uses large number of narrow-band subcarriers instead of a single wide-band subcarrier to transmit data over non-flat fading channels. This report begins with a refresher on Fourier transforms and uses them to explain the modulation scheme and characteristics of of OFDM like orthogonality and cyclic prefixes. A MATLAB implementation of OFDM was performed and the relevant code is attached at the end of the report.
\end{abstract}

\section{Introduction}
% The advent of internet and modern communication systems requires a need wide-band subcarriers to share data 

Orthogonal Frequency Division Multiplexing (OFDM) is a modulation technique that uses large number of narrow-band subcarriers instead of a single wide-band subcarrier to transmit data over non-flat fading channels.

\section{Fourier Transform}
A Fourier transform is a mathematical tool that translates functions of time or space into functions of temporal or spatial frequencies. Every signal in the world can be imagined as a sum of simple sinusoids at different frequencies.    

\subsection{Continuous Time Fourier Transform}
The Continuous (time) Fourier transform is a form of the Fourier transform that can be performed on continuous time signal, x(t). Examples of continuous systems are sine waves and analog sound signals. 

The Fourier Transform of a continuous signal can be determined using the formula:

\begin{equation}
    X(\omega) = \int_{-\infty}^{\infty} x(t)e^{-j\omega t} dt
\end{equation}
where $\omega \in (-\infty,+\infty)$

Which allows us to obtain the frequency domain form of our original signal.

\subsection{Discrete Time Fourier Transform}
The Discrete Fourier Transform is a form of the Fourier Transform that can be performed on Discrete time signals. Discrete time signals are signals that have a finite sequence of equally spaced samples, defined by a set sampling frequency. We often work with discrete time signals in digital signal processing. 

The Fourier Transform of a Discrete Time signal can be determined using the formula:
\begin{equation}
    X[\omega] = \Sigma_{n=-\infty}^{+\infty} x[k]e^{-j \omega k}
\end{equation}

where $\omega \in |-\pi,+\pi|$
\subsubsection{Fast Fourier Transform}
The Fast Fourier Transform is an algorithm that can determine either the Discrete Fourier transform of a given sequence or the Inverse Fourier transform of a sequence.

[Explain more about how the algorithm works]

\subsubsubsection{Cooley-Tukey algorithm}
- Commonly used FFT algorithm


\section{Orthogonal Frequency Division Multiplexing}


\begin{figure}
    \centering
    \includegraphics{}
    \caption{Caption}
    \label{fig:OFDM_Block_Diagram}
\end{figure}

\subsection{Orthogonality}

\subsection{Circular Convolution}


\section{OFDM Implementation}




\printbibliography
\end{document}
