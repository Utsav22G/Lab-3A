\documentclass{article}
\usepackage[utf8]{inputenc}
\usepackage{graphicx}
\usepackage{latexsym}
\usepackage{listings}
\usepackage{xcolor}
\usepackage{mathtools}
\usepackage{amssymb}
\usepackage{gensymb}
\usepackage{subfig}
\usepackage{mathrsfs}
\usepackage{amsmath}
\usepackage{biblatex}
\addbibresource{bib.bib}
\usepackage{float}
\usepackage{minted}
\definecolor{bg}{rgb}{0.95,0.95,0.95}


\title{Principles of Wireless Communications Lab 3A}
\author{Kristtiya Guerra, Utsav Gupta} %last name alphabetical
\date{\today}

\begin{document}

\maketitle

\begin{abstract}
    Orthogonal Frequency Division Multiplexing (OFDM) is a modulation technique that uses large number of narrow-band subcarriers instead of a single wide-band subcarrier to transmit data over non-flat fading channels. This report begins with a refresher on Fourier transforms and uses them to explain the modulation scheme and characteristics of of OFDM like orthogonality and cyclic prefixes. A MATLAB implementation of OFDM was performed and the relevant code is attached at the end of the report.
\end{abstract}

\section{Introduction}
% The advent of internet and modern communication systems requires a need wide-band subcarriers to share data 

Orthogonal Frequency Division Multiplexing (OFDM) is a modulation technique that uses large number of narrow-band subcarriers instead of a single wide-band subcarrier to transmit data over non-flat fading channels.

\section{Fourier Transform}
A Fourier transform is a mathematical tool that translates functions of time or space into functions of temporal or spatial frequencies. Every signal in the world can be imagined as a sum of simple sinusoids at different frequencies.    

\subsection{Continuous Time Fourier Transform}

\subsection{Discrete Time Fourier Transform}

\subsubsection{Fast Fourier Transform}

\subsubsection{Inverse Fast Fourier Transform}

\section{Orthogonal Frequency Division Multiplexing}

\begin{figure}
    \centering
    \includegraphics{}
    \caption{Caption}
    \label{fig:OFDM_Block_Diagram}
\end{figure}

\subsection{Orthogonality}

\subsection{Circular Convolution}


\section{OFDM Implementation}




\printbibliography
\end{document}